\documentclass{article}
\usepackage[utf8]{inputenc}
\usepackage{amsmath}
\usepackage{amsfonts}
\usepackage{graphicx}
\usepackage{graphicx,wrapfig,lipsum}


\begin{document}

\title{Clever Algebraic Tricks}
\author{Prajit Adhikari}
\date{}
\maketitle


In this article, we are going to see some beautiful and clever manipulation of algebraic problems in mathematical Olympiads. First, I expect the reader to know the basic algebraic formulae like $(a+c)^2, a^2-b^2,...$. Now, let's dig in:

Let's start with a problem.
\begin{enumerate}
    \item Factorise $(a-b)^3+(b-c)^3+(c-a)^3$.
\newline
First, let's think what can do we do to factorize this. Of course, expanding is first idea that comes to our mind. But, how about a beautiful approach, can you find it?
Hint: $(a-b)+(b-c)+(c-a)=0$ must tell you something.

Indeed, it is related to the expression:
$$a^3+b^3+c^3-3abc = (a+b+c)(a^2+b^2+c^2-ab-bc-ca)$$
Can you now draw the conclusion? Stop and take your time.

See that, if $a+b+c =0$, then we have $a^3+b^3+c^3-3abc=0$, which gives,
$$a^3+b^3+c^3=3abc$$
If you did think over it for sometime, then you must have found the same approach as below:
Since $(a-b)+(b-c)+(c-a)=0$, we can directly write 
$$(a-b)^3+(b-c)^3+(c-a)^3= 3(a-b)(b-c)(c-a)$$
As simple as that, and we are done. 


\item Let $a,b,c$ be distinct real numbers. Prove that 
$$\sqrt[3]{a-b} + \sqrt[3]{b-c} +\sqrt[3]{c-a}$$
can't be equal to zero.
\newline
Do you see something? What happens when it is zero? 
Assume for the sake of contradiction, that
$$\sqrt[3]{a-b} + \sqrt[3]{b-c} +\sqrt[3]{c-a} = 0$$
then, using, $a^3+b^3+c^3-3abc = (a+b+c)(a^2+b^2+c^2-ab-bc-ca)$, we get,
$$(a-b)+(b-c)+(c-a) = 3\sqrt[3]{(a-b)(b-c)(c-a)}$$
$$\implies (a-b)(b-c)(c-a)=0  \implies a=b, b=c, \textbf{ or } c=a$$
which is not possible since $a,b,c$ are distinct.
\end{enumerate}

Now, it's your turn to do some problems using this method:
\begin{enumerate}
    \item Prove that $\sqrt[3]{2- \sqrt{5}}+ \sqrt[3]{2+ \sqrt{5}}$ is a rational number.
    
    \item Let $x$ be a real number such that 
    $$ \sqrt[3]{x} +\frac{1}{\sqrt[3]{x}} =3$$
    Find the value of $x^3+\frac{1}{x^3}+2$.
    
    \item Find real solutions $x,y,z$ for the given equation
    $$x^3+y^3+z^3 = (x+y+z)^3$$
\end{enumerate}
I think this much should be enough to cover up this trick. Now, let's move on to another one.


Simon's Favorite Factoring Trick(SFFT)
SFFT is very basic yet intuitive way of solving problems. 









\end{document}